\documentclass[a4paper,12pt]{article}
\usepackage[utf8x]{inputenc}
\usepackage[russian]{babel}
\usepackage{amssymb}

%opening
\title{Асимптотика числа максимумов произведения линейных функций двух переменных}
\author{Ю.\,В.\,Чеканов}
\date{1986}

\newtheorem{law}{Теорема}
\newtheorem{lemm}{Лемма}

\begin{document}

\maketitle

\section{Оценки отношения чисел максимумов и минимумов}
  Назовём вещественный многочлен степени \(d\) от двух переменных невырожденным, если он имеет наибольшее возможное (равное \((d-1)^2\)) число вещественных критических точек. При больших \(d\) асимптотически половина этих точек будет сёдлами. Это вытекает из теоремы Эйлера и того, что индекс градиента многочлена на бесконечности не превышает \(O(d)\).

  Мы будем исследовать асимптотическое распределение остальных \mbox{\(\sim d^2/2\)} критических точек на максимумы и минимумы в случае, когда многочлен распадается в произведение линейных.

  \begin{law}
    \label{law:th1}
    Число максимумов невырожденного произведения линейных функций асимптотически не превосходит удвоенного числа минимумов:
    \begin{equation}
      \label{ineq:leq1}
      M \leqslant 2m + O(d),\qquad d \to \infty,
    \end{equation}
    причем оценка (\ref{ineq:leq1}) асимптотически достигается в том смысле, что для некоторой последовательности примеров
    \begin{equation}
      \label{ineq:leq2}
      M \geqslant 2m - O(d),\qquad d \to \infty.
    \end{equation}
  \end{law}

  Эквивалентная формулировка:

  \begin{law}
    \label{law:th2}
    Раскрасим области, на которые d прямых общего положения делят плоскость, в два цвета: белый и чёрный (соседние области окрашиваем в разные цвета). Тогда число чёрных областей асимптотически не превосходит удвоенного числа белых, и эта оценка асимптотически достигается.
  \end{law}

\section{Доказательство оценки сверху}
  Пусть \(d > 1\) и \(n_i\) число чёрных областей с \(i\) сторонами. Тогда \(2n_2+3n_3+\ldots=d^2\), так как \(d\) прямых общего положения делятся точками пересечения на \(d^2\) частей. Поскольку \(n_2 \leqslant d\), имеем для числа чёрных областей оценку \(n_2 + n_3 + \ldots \leqslant d(d+1)/3\), доказывающую оценку теоремы (\ref{law:th2}) (ибо всех областей \mbox{\(\sim d^2/2\)}).

\section{Пример, в котором чёрных областей почти вдвое больше, чем белых}
  В п.\,\ref{sec:ex} доказана принадлежащая Сильвестру \footnote{См.: Барр С.\,Как сажать деревья. --- В кн.: Математический цветник. М., 1983.}
  \begin{lemm}
    На плоскости можно расположить \(p\) точек так, чтобы асимптотически \mbox{\(\sim p^2/6\)} прямых (\(p \to \infty\)) проходили ровно через три точки.
  \end{lemm}

  Проективно-двойственное расположение состоит из \(p\) прямых, пересекающихся по три в \mbox{\(\sim p^2/6\)} точках. Выберем достаточно малое \(\epsilon\) и заменим каждую прямую двумя прямыми, отстоящими от неё на \(\epsilon\). Это~--- искомая конфигурация \(d=2p\) прямых.

  Раскрасим получившиеся области в черный и белый цвета. Каждой из \mbox{\(\sim p^2/3\)} областей исходной конфигурации \(p\) прямых соответствует черная область, каждому из \mbox{\(\sim p^2/2\)} отрезков~--- белая, и каждой точке тройного пересечения соответсвуют шесть черных треугольницов и один белый шестиугольник (топологически расположение в окрестности тройного пересечения стандартно, так как внутренний шестиугольник описан вокруг окружности радиуса \(\epsilon\)).

  Всего мы построили \(\sim p^2/3 + 6p^2/6 = d^2/3\) чёрных и \(\sim p^2/2 + p^2/6 = d^2/6\) белых областей, что и требовалось (\(d^2/3 + d^2/6 = d^2/2\), поэтому число неучтённых областей есть \(O(d)\)).

\section{Доказательство леммы}
  \label{sec:ex}
  Искомую конфигурацию образуют \(p\) точек вещественной циклической подгруппы плоской вещественной эллиптической кривой. Точнее, параметризуем нечётную ветвь вещественной эллиптической кривой интегралом первого рода, выбрав начало интегрирования в точке перегиба. По теореме Абеля три точки эллиптической кривой лежат на одной прямой, если и только если сумма значений интеграла (по модулю периодов) равна нулю. Точки, отвечающие значениям параметра \(k\omega/p\) (\(0 \leqslant k < p\)), где \(\omega\)~--- период (интеграл по всей ветви), образуют циклическую подгруппу \(\mathbb{Z}_p\) эллиптической кривой.

  Число неупорядоченных троек элементов группы \(\mathbb{Z}_p\), в сумме равных нулю, асимптотически \mbox{\(\sim p^2/6\)}. Действительно, упорядоченная тройка определяется парой элементов, и число разных упорядочений тройки отлично от шести лишь для \(O(p)\) троек. Поэтому построенные \(p\) точек лежат на \mbox{\(\sim p^2/6\)} прямых, что и требовалось.

  Автор приносит благодарность В.\,И.\,Арнольду за постановку задачи и внимание к работе, а также Д.\,Б.\,Фрадкину за полезные замечания.
\end{document}
